%!TEX TS-program = xelatex
%!TEX encoding = UTF-8 Unicode
\documentclass[UTF8]{ctexart}
\usepackage{geometry}                % See geometry.pdf to learn the layout options. There are lots.
\geometry{letterpaper}                   % ... or a4paper or a5paper or ... 
%\geometry{landscape}                % Activate for for rotated page geometry
%\usepackage[parfill]{parskip}    % Activate to begin paragraphs with an empty line rather than an indent
\usepackage{graphicx}
\usepackage{amssymb}

% Will Robertson's fontspec.sty can be used to simplify font choices.
% To experiment, open /Applications/Font Book to examine the fonts provided on Mac OS X,
% and change "Hoefler Text" to any of these choices.

\usepackage{fontspec,xltxtra,xunicode}
\defaultfontfeatures{Mapping=tex-text}
\setromanfont[Mapping=tex-text]{Hoefler Text}
\setsansfont[Scale=MatchLowercase,Mapping=tex-text]{Gill Sans}
\setmonofont[Scale=MatchLowercase]{Andale Mono}

\usepackage{url}
\usepackage[super,square]{natbib}

\setCJKmainfont[BoldFont={PingFangSC-Semibold}]{PingFangSC-Thin} % 设置中文字体,这里粗体使用苹方-Regular,默认使用苹方-Light


\usepackage{enumitem}



\usepackage{listings}
\usepackage{color}
\definecolor{mygreen}{rgb}{0,0.6,0}
\definecolor{mygray}{rgb}{0.5,0.5,0.5}
\definecolor{mymauve}{rgb}{0.58,0,0.82}
\lstset{
  backgroundcolor=\color{white},   % choose the background color
  basicstyle=\footnotesize\ttfamily,        % size of fonts used for the code
  breaklines=true,                 % automatic line breaking only at whitespace
  captionpos=b,                    % sets the caption-position to bottom
  commentstyle=\color{mygreen},    % comment style
  escapeinside={\%*}{*)},          % if you want to add LaTeX within your code
  keywordstyle=\color{blue},       % keyword style
  stringstyle=\color{mymauve},     % string literal style
  framextopmargin=50pt,
}


\title{RocketMQ学习笔记}
\author{亦心}
%\date{} 

\begin{document}
\maketitle

\newpage
\tableofcontents
\newpage


\section{初识}
\subsection{快速开始}

实验环境的搭建比较简单,详细步骤可以参考RocketMQ官网的Quick Start\cite{quick-start}。这里再补充几点:

\begin{enumerate}[itemindent=1em]
\item 直接下载使用binary release,避免maven “downloading the internet”。

\item 修改Name Server监听的端口号:
\begin{lstlisting}[language=sh]
> cat namesrv.properties
listenPort = 8876
> nohup sh bin/mqnamesrv -c namesrv.properties &
\end{lstlisting}
\item 修改Broker监听的端口号:
\begin{lstlisting}[language=sh]
> cat broker.properties
listenPort = 8874
> nohup sh bin/mqbroker -n localhost:8876 -c broker.properties &
\end{lstlisting}

\item 使用任意命令行工具前都需要保证环境变量已经导入:
\begin{lstlisting}[language=sh]
> export NAMESRV_ADDR=localhost:8876
\end{lstlisting}

\end{enumerate} 



\subsection{简单例子}
内容依然来自官网的Simple Example\cite{simple-example},为了能让例子运行,还几项配置:

\begin{enumerate}[itemindent=1em]

\item 在pom.xml中导入rocketmq-client对应的Maven依赖:
\begin{lstlisting}
<dependency>
  <groupId>com.alibaba.rocketmq</groupId>
  <artifactId>rocketmq-client</artifactId>
  <version>3.2.6</version>
</dependency>
\end{lstlisting}


\item 为producer设置Name Server的地址:
\begin{lstlisting}[language=Java]
producer.setNamesrvAddr("localhost:8876");
\end{lstlisting}

\end{enumerate} 


\subsection{消息类型}
RocketMQ支持3种消息模式:

\begin{enumerate}[itemindent=1em]
\item Reliable synchronous transmission 可靠同步传输。
\item Reliable asynchronous transmission 可靠异步传输。
\item One-way transmission One-way传输。
\end{enumerate} 

同步和异步传输都是通过send方法发送,其中异步传输还需要额外提供SendCallback接口的实现,而one-way消息通过sendOneway方法发送。

\subsection{Message类}
无论是那种传输模型,消息本身都储存在Message.java类中,在调用其构造函数时,至少需要传入消息主题、消息标签和消息体3个参数。以下是各个参数的含义:

\begin{enumerate}[itemindent=1em]
\item topic 消息主题,每一条消息都有且仅有一个主题。主题是生产者和消费者之间的桥梁。
\item flag 消息标志,系统不做干预,完全由应用决定如何使用。
\item body 消息体,即消息的内容。
\end{enumerate} 


以上3个参数直接存储在Message的字段中,以下3个参数以Key-Value的形式存储在系统属性字段中,各字段含义定义在MessageConst.java中,摘录如下:
\begin{enumerate}[itemindent=1em]
\item tag 消息标签,每一条消息只可以有一个标签。标签主要用于简单场景下的消息过滤\cite{filter-example}。
\item keys 消息关键词,查询消息使用,多个Key用KEY\_SEPARATOR隔开。
\item waitStoreMsgOK 是否等待服务器将消息存储完毕再返回,可能是等待刷盘完成或者等待同步复制到其他服务器。
\end{enumerate} 


\subsection{让消息飞一会儿}
跟着send方法一路走到sendDefaultImpl方法中,可以看到消息发送一共包含3个主要逻辑。

\subsubsection{TopicPublishInfo}
第一个主要逻辑,根据Topic选择TopicPublicInfo:

\section{测试}

\newpage
\bibliography{ref.bib}
\bibliographystyle{unsrt} 
\end{document}  