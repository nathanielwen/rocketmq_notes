\section{初识}

\subsection{快速开始}

实验环境的搭建比较简单,详细步骤可以参考RocketMQ官网的Quick Start\cite{quick-start}。这里再补充几点:

\begin{enumerate}[itemindent=1em]

\item 直接下载使用binary release,避免maven “downloading the internet”。

\item 修改Name Server监听的端口号:
\begin{lstlisting}[language=sh, style=customStyleBashLight]
> cat namesrv.properties
listenPort = 8876
> nohup sh bin/mqnamesrv -c namesrv.properties &
\end{lstlisting}

\item 修改Broker监听的端口号:
\begin{lstlisting}[language=sh, style=customStyleBashLight]
> cat broker.properties
listenPort = 8874
> nohup sh bin/mqbroker -n localhost:8876 -c broker.properties &
\end{lstlisting}

\item 使用任意命令行工具前都需要保证环境变量已经导入:
\begin{lstlisting}[language=sh, style=customStyleBashLight]
> export NAMESRV_ADDR=localhost:8876
\end{lstlisting}

\end{enumerate} 



\subsection{简单例子}
内容依然来自官网的Simple Example\cite{simple-example},为了能让例子运行,还几项配置:

\begin{enumerate}[itemindent=1em]

\item 在pom.xml中导入rocketmq-client对应的Maven依赖:
\begin{lstlisting}[style=customStyleHTMLLight]
<dependency>
  <groupId>com.alibaba.rocketmq</groupId>
  <artifactId>rocketmq-client</artifactId>
  <version>3.2.6</version>
</dependency>
\end{lstlisting}

\item 为producer设置Name Server的地址:
\begin{lstlisting}[language=Java, style=customStyleJavaLight]
producer.setNamesrvAddr("localhost:8876");

\end{lstlisting}

\end{enumerate} 


\subsection{消息类型}\label{subsection:message-type}
RocketMQ支持3种消息模式:

\begin{enumerate}[itemindent=1em]

\item Reliable synchronous transmission 可靠同步传输。

\item Reliable asynchronous transmission 可靠异步传输。

\item One-way transmission 单向传输。

\end{enumerate} 

同步和异步传输都是通过send方法发送,其中异步传输还需要额外提供SendCallback接口的实现,而单向消息通过sendOneway方法发送。

\subsection{Message类}
无论是那种传输模型,消息本身都储存在Message.java类中,在调用其构造函数时,至少需要传入消息主题、消息标签和消息体3个参数。以下是各个参数的含义:

\begin{enumerate}[itemindent=1em]

\item topic 消息主题,每一条消息都有且仅有一个主题。主题是生产者和消费者之间的桥梁。

\item flag 消息标志,系统不做干预,完全由应用决定如何使用。

\item body 消息体,即消息的内容。

\end{enumerate} 


以上3个参数直接存储在Message的字段中,以下3个参数以Key-Value的形式存储在系统属性字段中,各字段含义定义在MessageConst.java中,摘录如下:
\begin{enumerate}[itemindent=1em]

\item tag 消息标签,每一条消息只可以有一个标签。标签主要用于简单场景下的消息过滤\cite{filter-example}。

\item keys 消息关键词,查询消息使用,多个Key用空格隔开。

\item waitStoreMsgOK 是否等待服务器将消息存储完毕再返回,可能是等待刷盘完成或者等待同步复制到其他服务器。

\end{enumerate} 


\subsection{让消息飞一会儿}
跟着send方法一路走到sendDefaultImpl方法中,可以看到消息发送一共包含3个主要逻辑。

\subsubsection{TopicPublishInfo}

第一个主要逻辑,根据Topic选择TopicPublicInfo:
\begin{lstlisting}[language=Java, style=customStyleJavaLight]
TopicPublishInfo topicPublishInfo = this.tryToFindTopicPublishInfo(msg.getTopic());
\end{lstlisting}

具体实现时,先在本地的topicPublishInfoTable中查找一下是否缓存了这个topic的信息,如果没有,从NameServer查询,并更新到本地缓存中。其中的核心逻辑是把TopicRouteData转化为TopicPublishInfo:
\begin{lstlisting}[language=Java, style=customStyleJavaLight]
TopicPublishInfo publishInfo = topicRouteData2TopicPublishInfo(topic, topicRouteData);
\end{lstlisting}

这里涉及到的TopicPublishInfo类,一共包含4个字段:
\begin{enumerate}[itemindent=1em]
\item orderTopic 含义未明。
\item haveTopicRouterInfo 含义未明。
\item messageQueueList 消息队列列表,MessageQueue本身近似等价于Kafka的Partition(只是近似,具体区别下文有详细说明),每个queue都记录了消息主题、所属的broker、以及queueId。
\item sendWhichQueue 用来判断发送到哪个队列。
\end{enumerate}

使用mqadmin的topicRoute命令可以查看某个topic的路由信息:
\begin{lstlisting}[language=sh, style=customStyleBashLight]
> sh bin/mqadmin topicRoute -t TopicName

{
    "brokerDatas":[
        {
            "brokerAddrs":{0:"ip-a:8861"},
            "brokerName":"broker-b",
            "cluster":"DefaultCluster"
        },
        {
            "brokerAddrs":{0:"ip-b:8874"},
            "brokerName":"broker-a",
            "cluster":"DefaultCluster"
        }
    ],
    "filterServerTable":{},
    "queueDatas":[
        {
            "brokerName":"broker-a",
            "perm":6,
            "readQueueNums":4,
            "topicSynFlag":0,
            "writeQueueNums":4
        },
        {
            "brokerName":"broker-b",
            "perm":6,
            "readQueueNums":4,
            "topicSynFlag":0,
            "writeQueueNums":4
        }
    ]
}
\end{lstlisting}

其中的核心逻辑是设置messageQueueList,遍历每个QueueData(记为qd):

\begin{enumerate}[itemindent=1em]
\item 根据perm属性判断这个Queue是否可写,如果可写则继续,否则放弃这个qd。
\item 找到与这个qd的brokerName相同的brokerData(记为bd),如果存在则继续,否则放弃这个qd。
\item bd的brokerAddrs信息,其实是一个Map,key是编号,value是ip+port,编号为0的Broker是Master Broker,如果存在Master Broker则继续,否则放弃这个qd。
\item 根据qd的writeQueueNums信息生成MessageQueue对象。
\end{enumerate}


\subsubsection{MessageQueue}
第二个主要逻辑,选择MessageQueue:

\begin{lstlisting}[language=Java, style=customStyleJavaLight]
MessageQueue tmpmq = topicPublishInfo.selectOneMessageQueue(lastBrokerName);
\end{lstlisting}

所谓选择MessageQueue实际上也就是设置消息的负载均衡策略,默认的策略是round-robin的:

\begin{enumerate}[itemindent=1em]
\item 如果lastBrokerName为空,则根据某个计数器对messageQueueList.size()取模判断发送到哪个队列中去。
\item 如果lastBrokerName为不空,则除了要实现以上逻辑,还需要保证选取的broker不能是lastBroker。
\end{enumerate}

如果要选择其他的策略,需要使用指定MessageQueueSelector的发送方式\cite{order-example},该接口目前有三种实现模式:SelectMessageQueueByHash、SelectMessageQueueByMachineRoom以及SelectMessageQueueByRandoom。


\subsubsection{SendResult}
第三个主要逻辑,发送消息:
\begin{lstlisting}[language=Java, style=customStyleJavaLight]
sendResult = this.sendKernelImpl(msg, mq, communicationMode, sendCallback, timeout);
\end{lstlisting}

其中最核心的步骤是调用MQClientAPIImpl的sendMessage方法。上文 \ref{subsection:message-type} 提到,消息的发送分为同步、异步和单向3种模式。体现在代码里就是:
\begin{lstlisting}[language=Java, style=customStyleJavaLight]
switch (communicationMode) {
    case ONEWAY:
        this.remotingClient.invokeOneway(addr, request, timeoutMillis);
        return null;
    case ASYNC:
        this.sendMessageAsync(addr, brokerName, msg, timeoutMillis, request, sendCallback);
        return null;
    case SYNC:
        return this.sendMessageSync(addr, brokerName, msg, timeoutMillis, request);
    default:
        assert false;
        break;
    }
\end{lstlisting}

消息发送在底层是通过netty实现的,三种模式的区别在于对ChannelFutureListener的实现:
\begin{enumerate}[itemindent=1em]
\item invokeOneway 发送消息,不设置回调,不处理回执:调用RemotingClient.invokeOneway方法,其中ChannelFutureListener只是释放了信号量并且对于请求失败的场景打印了warning日志,也不会调用MQClientAPIImpl.processSendResponse方法。
\item sendMessageSync 发送消息,不设置回调,阻塞、同步的处理回执:调用RemotingClient.invokeSync方法,其中ChannelFutureListener对于ChannelFuture的返回结果是否成功做了不同的业务处理,并调用ResponseFuture的waitResponse方法进入阻塞状态,随后调用MQClientAPIImpl.processSendResponse方法。
\item sendMessageAsync 发送消息,设置回调,非阻塞、异步的处理回执:调用RemotingClient.invokeAsync方法,其中ChannelFutureListener的逻辑与同步模式基本类似,同时还需要实现InvokeCallback接口,该接口在ChannelFuture返回失败时会被调用到。
\end{enumerate}

目前对于netty的了解比较局限,从源码分析来看,netty对于请求的响应都是异步处理的,具体还可以拆分成2个逻辑:
\begin{enumerate}[itemindent=1em]
\item NettyRemotingClient的内部类NettyClientHandler的channelRead0方法。该方法根据是否设置了SendCallback可以区分出异步模式和非异步模式。同时,该方法通过调用ResponseFuture的setResponseCommand方法,使得前述同步模式的阻塞得以解除。
\item ChannelFutureListener的operationComplete方法。该方法应该是channelRead0方法完成之后的回调操作。
\end{enumerate}

% 信号量实现仅发送1次。




\subsection{和Kafka的对比}
假设基础环境都是4台物理机\{A, B, C, D\}和4个Topic。

\subsubsection{RocketMQ}
2Master,2Slave模式下,Broker-A=\{master: A, slave: B\},Broker-B=\{master: C, slave: D\},假设Broker-A和Broker-B各设置了4个Message Queue(writeQueueNums为4),则每个Topic可以往8个Message Queue上写数据。Broker-\{A,B\}各存储1/2的数据。

\subsubsection{Kafka}
Broker-A=\{A\},Broker-B=\{B\},Broker-C=\{C\},Broker-D=\{D\},假设A的Replica在B上,B的Replica在A上,C的Replica在D上,D的Replica在C上;每个Topic分为8个Partition则Broker-\{A,B,C,D\}理论上应该各存储1/4的数据。
