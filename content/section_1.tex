\section{初识}
\subsection{快速开始}

实验环境的搭建比较简单,详细步骤可以参考RocketMQ官网的Quick Start\cite{quick-start}。这里再补充几点:

\begin{enumerate}[itemindent=1em]
\item 直接下载使用binary release,避免maven “downloading the internet”。

\item 修改Name Server监听的端口号:
\begin{lstlisting}[language=sh, style=customStyleBashDark]
> cat namesrv.properties
listenPort = 8876
> nohup sh bin/mqnamesrv -c namesrv.properties &
\end{lstlisting}
\item 修改Broker监听的端口号:
\begin{lstlisting}[language=sh, style=customStyleBashDark]
> cat broker.properties
listenPort = 8874
> nohup sh bin/mqbroker -n localhost:8876 -c broker.properties &
\end{lstlisting}

\item 使用任意命令行工具前都需要保证环境变量已经导入:
\begin{lstlisting}[language=sh, style=customStyleBashDark]
> export NAMESRV_ADDR=localhost:8876
\end{lstlisting}

\end{enumerate} 



\subsection{简单例子}
内容依然来自官网的Simple Example\cite{simple-example},为了能让例子运行,还几项配置:

\begin{enumerate}[itemindent=1em]

\item 在pom.xml中导入rocketmq-client对应的Maven依赖:
\begin{lstlisting}[style=customStyleHTMLDark]
<dependency>
  <groupId>com.alibaba.rocketmq</groupId>
  <artifactId>rocketmq-client</artifactId>
  <version>3.2.6</version>
</dependency>
\end{lstlisting}


\item 为producer设置Name Server的地址:
\begin{lstlisting}[language=Java, style=customStyleJavaDark]
producer.setNamesrvAddr("localhost:8876");
\end{lstlisting}

\end{enumerate} 


\subsection{消息类型}
RocketMQ支持3种消息模式:

\begin{enumerate}[itemindent=1em]
\item Reliable synchronous transmission 可靠同步传输。
\item Reliable asynchronous transmission 可靠异步传输。
\item One-way transmission One-way传输。
\end{enumerate} 

同步和异步传输都是通过send方法发送,其中异步传输还需要额外提供SendCallback接口的实现,而one-way消息通过sendOneway方法发送。

\subsection{Message类}
无论是那种传输模型,消息本身都储存在Message.java类中,在调用其构造函数时,至少需要传入消息主题、消息标签和消息体3个参数。以下是各个参数的含义:

\begin{enumerate}[itemindent=1em]
\item topic 消息主题,每一条消息都有且仅有一个主题。主题是生产者和消费者之间的桥梁。
\item flag 消息标志,系统不做干预,完全由应用决定如何使用。
\item body 消息体,即消息的内容。
\end{enumerate} 


以上3个参数直接存储在Message的字段中,以下3个参数以Key-Value的形式存储在系统属性字段中,各字段含义定义在MessageConst.java中,摘录如下:
\begin{enumerate}[itemindent=1em]
\item tag 消息标签,每一条消息只可以有一个标签。标签主要用于简单场景下的消息过滤\cite{filter-example}。
\item keys 消息关键词,查询消息使用,多个Key用KEY\_SEPARATOR隔开。
\item waitStoreMsgOK 是否等待服务器将消息存储完毕再返回,可能是等待刷盘完成或者等待同步复制到其他服务器。
\end{enumerate} 


\subsection{让消息飞一会儿}
跟着send方法一路走到sendDefaultImpl方法中,可以看到消息发送一共包含3个主要逻辑。

\subsubsection{TopicPublishInfo}

第一个主要逻辑,根据Topic选择TopicPublicInfo:
\begin{lstlisting}[language=Java, style=customStyleJavaDark]
TopicPublishInfo topicPublishInfo = this.tryToFindTopicPublishInfo(msg.getTopic());
\end{lstlisting}

具体实现时,先在本地的topicPublishInfoTable中查找一下是否缓存了这个topic的信息,如果没有,从NameServer查询,并更新到本地缓存中。其中的核心逻辑是把TopicRouteData转化为TopicPublishInfo:
\begin{lstlisting}[language=Java, style=customStyleJavaDark]
TopicPublishInfo publishInfo = topicRouteData2TopicPublishInfo(topic, topicRouteData);
\end{lstlisting}

这里涉及到的TopicPublishInfo类,一共包含4个字段:
\begin{enumerate}[itemindent=1em]
\item orderTopic 含义未明。
\item haveTopicRouterInfo 含义未明。
\item messageQueueList 消息队列列表,MessageQueue本身近似等价于Kafka的Partition(只是近似,具体区别下文有详细说明),每个queue都记录了消息主题、所属的broker、以及queueId。
\item sendWhichQueue 用来判断发送到哪个队列。
\end{enumerate}

使用mqadmin的topicRoute命令可以查看某个topic的路由信息:
\begin{lstlisting}[language=sh, style=customStyleBashDark]
> sh bin/mqadmin topicRoute -t TopicName

{
    "brokerDatas":[
        {
            "brokerAddrs":{0:"ip-a:8861"},
            "brokerName":"broker-b",
            "cluster":"DefaultCluster"
        },
        {
            "brokerAddrs":{0:"ip-b:8874"},
            "brokerName":"broker-a",
            "cluster":"DefaultCluster"
        }
    ],
    "filterServerTable":{},
    "queueDatas":[
        {
            "brokerName":"broker-a",
            "perm":6,
            "readQueueNums":4,
            "topicSynFlag":0,
            "writeQueueNums":4
        },
        {
            "brokerName":"broker-b",
            "perm":6,
            "readQueueNums":4,
            "topicSynFlag":0,
            "writeQueueNums":4
        }
    ]
}
\end{lstlisting}

其中的核心逻辑是设置messageQueueList:
遍历每个QueueData(记为qd),

\begin{enumerate}[itemindent=1em]
\item 根据perm属性判断这个Queue是否可写,如果可写则继续,否则放弃这个qd。
\item 找到与这个qd的brokerName相同的brokerData(记为bd),如果存在则继续,否则放弃这个qd。
\item bd的brokerAddrs信息,其实是一个Map,key是编号,value是ip+port,编号为0的Broker是Master Broker,如果存在Master Broker则继续,否则放弃这个qd。
\item 根据qd的writeQueueNums信息生成MessageQueue对象。
\end{enumerate}
